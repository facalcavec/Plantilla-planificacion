\documentclass[11pt]{charter}

% El títulos de la memoria, se usa en la carátula y se puede usar el cualquier lugar del documento con el comando \ttitle
\titulo{Cargador de Baterías Modular con Monitoreo Remoto} 

% Nombre del posgrado, se usa en la carátula y se puede usar el cualquier lugar del documento con el comando \degreename
\posgrado{Carrera de Especialización en Sistemas Embebidos} 
%\posgrado{Carrera de Especialización en Internet de las Cosas} 
%\posgrado{Carrera de Especialización en Intelegencia Artificial}
%\posgrado{Maestría en Sistemas Embebidos} 
%\posgrado{Maestría en Internet de las cosas}

% Tu nombre, se puede usar el cualquier lugar del documento con el comando \authorname
\autor{Felipe A. Calcavecchia} 

% El nombre del director y co-director, se puede usar el cualquier lugar del documento con el comando \supname y \cosupname y \pertesupname y \pertecosupname
\director{Nombre del Director}
\pertenenciaDirector{pertenencia} 
% FIXME:NO IMPLEMENTADO EL CODIRECTOR ni su pertenencia
\codirector{} % si queda vacio no se deberíá incluir 
\pertenenciaCoDirector{}

% Nombre del cliente, quien va a aprobar los resultados del proyecto, se puede usar con el comando \clientename y \empclientename
\cliente{Luis A. Rosende}
\empresaCliente{\textit{\textbf{proba}} Baterías}

% Nombre y pertenencia de los jurados, se pueden usar el cualquier lugar del documento con el comando \jurunoname, \jurdosname y \jurtresname y \perteunoname, \pertedosname y \pertetresname.
\juradoUno{Nombre y Apellido (1)}
\pertenenciaJurUno{pertenencia (1)} 
\juradoDos{Nombre y Apellido (2)}
\pertenenciaJurDos{pertenencia (2)}
\juradoTres{Nombre y Apellido (3)}
\pertenenciaJurTres{pertenencia (3)}
 
\fechaINICIO{27 de junio de 2020}		%Fecha de inicio de la cursada de GdP \fechaInicioName
\fechaFINALPlanificacion{22 de Agosto de 2020} 	%Fecha de final de cursada de GdP
\fechaFINALTrabajo{5 de diciembre de 2020}		%Fecha de defensa pública del trabajo final

\usepackage{enumitem}

\begin{document}

\maketitle
\thispagestyle{empty}
\pagebreak


\thispagestyle{empty}
{\setlength{\parskip}{0pt}
\tableofcontents{}
}
\pagebreak


\section{Registros de cambios}
\label{sec:registro}


\begin{table}[ht]
\label{tab:registro}
\centering

\begin{tabularx}{\linewidth}{@{}|c|X|c|@{}}
\hline
\rowcolor[HTML]{C0C0C0} 
Revisión & \multicolumn{1}{c|}{\cellcolor[HTML]{C0C0C0}Detalles de los cambios realizados} & Fecha      \\ \hline
1.0      & Creación del documento                                                          & 27/06/2020 \\ \hline
1.1      & Se completó Propósito, Alcance, Supuestos, Requerimientos, Entregables y Desglose del trabajo en taréas 																						   & 10/07/2020 \\ \hline
%1.2      & Otro ejemplo \newline
%		   Con texto partido \newline
%		   En varias líneas \newline
%		   A propósito                                                                     %& dd/mm/aaaa \\ \hline
\end{tabularx}
\end{table}

\pagebreak



\section{Acta de Constitución del Proyecto}
\label{sec:acta}

\begin{flushright}
Buenos Aires, \fechaInicioName
\end{flushright}

\vspace{2cm}

Por medio de la presente se acuerda con el Ing. \authorname\hspace{1px} que su Trabajo Final de la \degreename\hspace{1px} se titulará ``\ttitle'', consistirá esencialmente en el prototipo preliminar de una fuente utilizada como cargador y una placa de control que garantice la funcionalidad del mismo, y tendrá un presupuesto preliminar estimado de 600 hs de trabajo y {\$60.000}, con fecha de inicio \fechaInicioName\hspace{1px} y fecha de presentación pública \fechaFinalName.

Se adjunta a esta acta la planificación inicial.

\vfill

% Esta parte se construye sola con la información que hayan cargado en el preámbulo del documento y no debe modificarla
\begin{table}[ht]
\centering
\begin{tabular}{ccc}
\begin{tabular}[c]{@{}c@{}}Ariel Lutenberg \\ Director posgrado FIUBA\end{tabular} &  & \begin{tabular}[c]{@{}c@{}}\clientename \\ \empclientename \end{tabular} \vspace{2.5cm} \\ 
\multicolumn{3}{c}{\begin{tabular}[c]{@{}c@{}} \supname \\ Director del Trabajo Final\end{tabular}} \vspace{2.5cm} \\
\begin{tabular}[c]{@{}c@{}}\jurunoname \\ Jurado del Trabajo Final\end{tabular}     &  & \begin{tabular}[c]{@{}c@{}}\jurdosname\\ Jurado del Trabajo Final\end{tabular}  \vspace{2.5cm}  \\
\multicolumn{3}{c}{\begin{tabular}[c]{@{}c@{}} \jurtresname\\ Jurado del Trabajo Final\end{tabular}} \vspace{.5cm}                                                                     
\end{tabular}
\end{table}




\section{Descripción técnica-conceptual del Proyecto a realizar}
\label{sec:descripcion}

Hoy en día las baterías de uso industrial constituyen una parte fundamental en sistemas de respaldo de alimentación, vehículos de tracción eléctrica y otros multiples usos. Su elevado costo respecto de los  dispositivos que alimentan, hacen que un buen uso y mantenimiento sea una cuestión a tener en cuenta a la hora de considerar su vida útil.
\\[3pt]
Todo esto lleva a la necesidad de desarrollar un cargador que asegure una carga adecuada (en tensión y corriente), de acuerdo al tipo de batería, a su estado de carga y a las condiciones ambientales que la rodean.
\\[3pt]
Si tenemos en cuenta que el mercado local ofrece algunos cargadores nacionales, la mayoría se basan en tecnologías antiguas. Por el lado de los importados, tecnológicamente mas avanzados, necesitan certificaciones aduaneras que elevan su precio.
\\[3pt]
Si bien la empresa produce cargadores, estos son un complemento en la comercialización de baterías para automoviles. Con este proyecto se pretende aumentar la presencia de la companía en el segmento del mercado que corresponde a las baterías industriales expandiendo su modelo de negocio a otras áreas menos explotadas.\\ 
En la Figura \ref{fig:modeloCanvas} se puede observar el modelo Canvas de negocio.

\begin{figure}[h]
\centering 
\includegraphics[width=.9\textwidth]{./Figuras/Canvas1.pdf}
\caption{Modelo Canvas de negocio}
\label{fig:modeloCanvas}
\end{figure}

El presente proyecto se destaca en tres aspectos que le agregan valor, dos enfocados en el consumidor final y uno en el cliente.
\\[3pt]
En lo pertinente al consumidor final, y desde el punto de vista del hardware, se descartan los voluminosos y pesados transformadores por fuentes conmutadas de alta frecuencia, mas eficaces, pequeñas y livianas.\\
Desde el lado del firmware de control, este, permite adaptarse a cada tipo de banco de baterías en forma particular. Además como aspecto innovador, se incorpora un monitoreo de cada una de las cargas que realiza y con esa información genera un log que permite hacer un análisis periódico del estado de la batería y activar alarmas tempranas en caso de detectar alguna anomalía.
\\[3pt]
Por el lado del cliente, se beneficia al disminuir stock inmovilizado, teniendo un solo modelo de cargador modularizado y configurable, en vez de varios cargadores, uno por cada tipo de batería, simplificando su producción y ahorrando costos.

En la Figura \ref{fig:diagBloques} se muestra el diagrama en bloques del proyecto a realizar. Se observa que el cargador posee una disposición modular en la que admite 1, 2 ó 3 fuentes. Cada una puede aportar hasta 40 Amperes. Esto posiblita configurar el cargador para adaptarse a los requerimientos de los distintos tipos de baterías, abarcando las tensiones standars más utilizadas (12V, 24V, 36V y 48V) y corrientes de carga que pueden ir desde 1A hasta 120A.\\[3pt]
La placa que controla al cargador contiene un microprocesador capaz de suministrar tres señales PWM independientes  que manejan las tensiones de salidas de las tres fuentes, cinco canales ADC para leer sensores, un puerto con entradas/salidas digitales para el display, teclado, relés de alarma y otros accesorios como indicadores luminosos, ventiladores, etc, y por último una comunicacion serie para los modulos de WIFI y el reloj de tiempo real.\\[3pt]
El firmware controla, en forma secuencial, cuatro etapas de carga:

%\vspace{5px}

\begin{figure}[htpb]
\centering 
\includegraphics[width=.8\textwidth]{./Figuras/Plan_Figura1.pdf}
\caption{Diagrama en bloques del sistema}
\label{fig:diagBloques}
\end{figure}

%\vspace{5px}

\begin{enumerate}
\item CARGA A FONDO: Suministra aproximadamente el 80\% de la carga total, se realiza a corriente constante y se registra el tiempo de duración.
\item CARGA POR ABSORCION: Le entrega el 20\% restante de carga y se realiza a tensión constante. Dura aproximadamente el mismo tiempo que la carga a Fondo.
\item CARGA A FLOTE: Cuando finaliza la carga, se fija una tensión y corriente máxima de forma que la batería pueda quedar conectada al cargador indefinidamente sin provocar sobrecargas.
\item ECUALIZACIÓN: Cada un número determinado de cargas, se realiza este paso para emparejar los elementos que conforman la batería. Se hace forzando una sobrecarga durante un tiempo controlado relativamente corto.
\end{enumerate}

Las corrientes de las fuentes son medidas y comparadas con las de referencia, generando una señal de error que se usa para actuar sobre los PWM’s formando un sistema de lazo cerrado que se controla por un algoritmo PID.
Al finalizar la carga, se genera un registro identificando parámetros como, tensión, corriente, fecha y hora de inicio y finalización, Ampere-Hora suministrado, temperatura y cantidad de cargas realizadas. Estos datos sirven para llevar una “historia clínica” de la batería y verificar si existe alguna anomalía para activar las correspondientes alarmas.

%\clearpage
%\newpage

\section{Identificación y análisis de los interesados}
\label{sec:interesados}



\begin{table}[ht]
%\caption{Identificación de los interesados}
%\label{tab:interesados}
\begin{tabularx}{\linewidth}{@{}|l|X|X|l|@{}}
\hline
\rowcolor[HTML]{C0C0C0} 
Rol           & Nombre y Apellido & Organización 	   & Puesto 	\\ \hline
Auspiciante   &                   &              	   &        	\\ %\hline
Cliente       & \clientename      &\empclientename	   & Dto. Ventas       	\\ %\hline
Impulsor      &                   &              	   &        	\\ \hline
			  &					  &					   &			\\
Responsable   & \authorname       & \textbf{\textit{proba}} Instrumentos       	   & Ing. Desarrollo	\\ 
			  &					  &					   &			\\ \hline
			  &					  &					   &			\\			  
Equipo		  & Luca Calcavecchia & UTN-FRH       	   & Alumno    	\\ 
			  &					  &					   &			\\ \hline
			  &					  &					   &			\\			  
Orientador    & \supname	      & \pertesupname 	   & Director	Trabajo final \\       			  &					  &					   &			\\ \hline
%Equipo        & miembro1 \newline 
%				miembro2          &              	   &        	\\ \hline
%Opositores    &                   &              	   &        	\\ \hline
			  &					  &					   &			\\
Usuario final & Empresas		  & \hspace{2cm}-	   & \hspace{1.8cm}-			\\ 
			  &					  &					   &			\\ \hline
\end{tabularx}
\end{table}

\begin{itemize}
\item[•] El Auspiciante, Cliente e Impulsor es el titular de \textbf{\textit{proba}} Baterías y socio con el Responsable del proyecto en \textbf{\textit{proba}} Instrumentos.
\item[•] El Equipo se encarga del diseño del gabinete y la documentación correspondiente.
\end{itemize}


\section{1. Propósito del proyecto}
\label{sec:proposito}

\begin{consigna}{black}
El propósito de este proyecto es diseñar e implementar un prototipo funcional de un cargador de baterías modular, aplicando los conocimiento que se van adquiriendo en el curso. A su vez que esos concimientos sirvan para ser incorporados como metodología de trabajo a futuros productos realizados en la empresa.
\end{consigna}

\section{2. Alcance del proyecto}
\label{sec:alcance}

Este proyecto incluirá el diseño y construcción de un prototipo funcional de un cargador de baterías que conste de un sistema embebido formado por una placa, que interactúe con sus perisféricos y controle las fuentes de carga. También  se incluirá toda la documentación y archivos necesarios para su producción, como su manual de uso e instalación.

No queda incluido en el presente proyecto, el diseño y construcción de las fuentes, el diseño del gabinete y partes mecánicas. Tampoco incluye la implementación de una aplicación de software para la lectura remota del reporte de cargas. Solo se enviarán los datos crudos del log.


\section{3. Supuestos del proyecto}
\label{sec:supuestos}

Para el desarrollo del presente proyecto se supone que: 

\begin{enumerate}
\item Se disponerá de la información de marketing de los posibles usuarios para analizar las funcionalidades del proyecto.
\item Se contará con los suficientes recursos económicos para la compra de todo el material necesario.
\item Se supone que los componentes a utilizar se consiguen localmente o que se dispone de todos los requisitos necesarios para su importación en caso de ser necesario.
\item Se supone que las fuentes a utilizar tienen el correspondiente certificado de seguridad eléctrica.
\item Se supone que los tiempos de importación están dentro de lo planificado.
\item Se supone que los tiempos de fabricación están dentro de lo planificado.
\item Se supone que se tendrá acceso a las instalaciones y elementos necesarios para realizar las pruebas de campo.
\item Debido a que las pruebas de campo son de larga duración, se dispondrá del tiempo necesario para usar las instalaciones sin restricciones.

\end{enumerate}


\section{4. Requerimientos}
\label{sec:requerimientos}

Los requerimientos del presente proyecto se establecieron luego de acordar con el cliente y muchos a sugerencia de posibles usuarios finales. Los mismos se describen a continuación en orden prioritario.


\begin{enumerate}
	\item \textbf{Requerimientos generales del proyecto}
	\begin{enumerate}[label*=\arabic*.]
		\item Fecha de entrega del proyecto terminado: 5 de Julio de 2021
		\item El responsable asegura al cliente el know-how del proyecto.
		\item Podrá alimentarse con línea de red monofásica o trifásica.
		\item Deberá contemplar protecciones de alimentación atraves de llaves térmicas. 
	\end{enumerate}

	\item \textbf{Requerimientos funcionales}
	\begin{enumerate}[label*=\arabic*.]
		\item \textbf{Requerimientos de Hardware}
			\begin{enumerate}[label*=\arabic*.]
				\item El dispositivo debe contemplar un diseño modular.
				\item El diseño modular debe permitir su reconfiguración.
				\item Debe tener un teclado accesible para su configuración y manejo.
				\item Debe poseer un display que permita visualizar la configuración y parámetros mensurables.
				\item Debe poseer indicadores luminosos bien visibles.					
				\item Cada fuente debe tener su propio sensor de corriente.
				\item El sensor de tensión es común a todas las fuentes.
				\item Se agrega un botón de parada de emergencia.			
			\end{enumerate}
		\item \textbf{Requerimientos de Firmware}
			\begin{enumerate}[label*=\arabic*.]
				\item Debe permitir configurar la tensión y corriente máxima de carga.
				\item Debe permitir configurar los tiempos máximos para cada etapa de carga.
				\item Debe guardar al menos dos configuraciones.
				\item Tendrá que medir tensión, corriente y temperatura.
				\item Tendrá que garantizar cuatro etapas de carga:
					\begin{enumerate}[label*=\arabic*.]
						\item Carga a Fondo.
						\item Carga por Absorción.
						\item Carga a Flote.
						\item Ecualización.
					\end{enumerate}
				\item Tendrá un algoritmo de parada de emergencia.
				\item El control de carga se realizará por un algoritmo PID
				\item Se registrará la fecha y hora de inicio y finalización de cada carga a través de un RTC.
				\item Debe guardar las últimas mil cargas realizadas.
				\item Con los datos recavados se podrá determinar anomalías y generar alarmas. 
				\item El registro de datos almacenados debe estar disponible para ser consultado remotamente.
			\end{enumerate}					
	\end{enumerate}
	\item \textbf{Requerimientos no funcionales}
	\begin{enumerate}[label*=\arabic*.]
		\item Se deberá generar documentación:
			\begin{enumerate}[label*=\arabic*.]
				\item Esquematicos eléctricos.
				\item Manual de instalación.
				\item Manual de uso.
			\end{enumerate}
		\item Se contará con la correspondiente certificación eléctrica otorgada por un laboratorio habilitado.
		\item El grado de protección del sistema debe ser como mínimo IP50.
		\item Se debe garantizar un servicio de post venta por al menos 5 años.
	\end{enumerate}
\end{enumerate}


\section{5. Entregables principales del proyecto}
\label{sec:entregables}


\begin{itemize}
\item Plan de trabajo
\item Memoria del proyecto
\item Prototipo funcional
\item Diagrama esquemático
\item Manual de instalación
\item Manual de uso
\item Código fuente
 
\end{itemize}

\section{6. Desglose del trabajo en tareas}
\label{sec:wbs}

\begin{enumerate}
	\item \textbf{Planificación del proyecto (20hs)}
	\begin{enumerate}[label*=\arabic*.]
		\item Definición de requerimientos con el cliente (5hs)
		\item Confección del plan de trabajo (15hs)
	\end{enumerate}

	\item \textbf{Investigación preliminar (65hs)}
	\begin{enumerate}[label*=\arabic*.]
		\item Información sobre la competencia (5hs)
		\item Análisis técnico-económico (10hs)
		\item Análisis comparativo para determinar que microprocesador usar (10hs)	
		\item Investigación sobre los perisféricos a utilizar (15hs)
		\item Estudio de los datasheet de los periféricos elegidos (25hs)   
	\end{enumerate}

	\item \textbf{Desarrollo del Hardware (140hs)}
	\begin{enumerate}[label*=\arabic*.]
		\item Diseño del diagrama esquemático (30hs)
		\item Diseño del diagrama de conexión (20hs)
		\item Diseño de los PCB's preliminares (30hs)	
		\item Armado de los PCB's  (10hs)
		\item Ensamblado del prototipo inicial (10hs)
		\item Pruebas y ensayos preliminares (40hs)   
	\end{enumerate}
	
	\item \textbf{Desarrollo del Firmware (250hs)}
	\begin{enumerate}[label*=\arabic*.]
		\item Definición de funciones a realizar (30hs)
		\item Modularización del código
			\begin{enumerate}[label*=\arabic*.]
				\item Módulo de inicialización (10hs)
				\item Módulo de teclado (10hs)
				\item Módulo del menú y presentación (30hs)
				\item Módulo de configuración (40hs)
				\item Módulo de adquisición de datos (25hs)
				\item Módulo PID de control (20hs)	
				\item Módulo de comunicación con perisféricos (30hs)
				\item Módulo para guardar los registros  (20hs)
				\item Módulo de diagnóstico y alarmas (35hs)   
			\end{enumerate}	  
	\end{enumerate}
	
	\item \textbf{Vereficación y validación (90hs)}
	\begin{enumerate}[label*=\arabic*.]
		\item Integración del sistema (20hs)
		\item Pruebas de campo (40hs)
		\item Correcciones de los PCB's (10hs)	
		\item Busqueda de posibles bugs  (10hs)
		\item Ensamblado final del prototipo (10hs)  
	\end{enumerate}	
	
	\item \textbf{Proceso de cierre (95hs)}
	\begin{enumerate}[label*=\arabic*.]
		\item Elaboración de la documentación y manuales (20hs)
		\item Elaboración de la memoria técnica (60hs)
		\item Preparación de la presentación final (15hs)	 
	\end{enumerate}		
\end{enumerate}

\textbf{Cantidad total de horas: (660 hs)}


\section{7. Diagrama de Activity On Node}
\label{sec:AoN}

\begin{consigna}{red}
Armar el AoN a partir del WBS definido en la etapa anterior. 

%La figura \ref{fig:AoN} fue elaborada con el paquete latex tikz y pueden consultar la siguiente referencia \textit{online}:

%\url{https://www.overleaf.com/learn/latex/LaTeX_Graphics_using_TikZ:_A_Tutorial_for_Beginners_(Part_3)\%E2\%80\%94Creating_Flowcharts}

\end{consigna}

\begin{figure}[htpb]
\centering 
\includegraphics[width=.8\textwidth]{./Figuras/AoN.png}
\caption{Diagrama en \textit{Activity on Node}}
\label{fig:AoN}
\end{figure}

Indicar claramente en qué unidades están expresados los tiempos.
De ser necesario indicar los caminos semicríticos y analizar sus tiempos mediante un cuadro.
Es recomendable usar colores y un cuadro indicativo describiendo qué representa cada color, como se muestra en el siguiente ejemplo:



\section{8. Diagrama de Gantt}
\label{sec:gantt}

\begin{consigna}{red}
Utilizar el software Gantter for Google Drive o alguno similar para dibujar el diagrama de Gantt.

Existen muchos programas y recursos \textit{online} para hacer diagramas de gantt, entre las cuales destacamos:

\begin{itemize}
\item Planner
\item GanttProject
\item Trello + \textit{plugins}. En el siguiente link hay un tutorial oficial: \\ \url{https://blog.trello.com/es/diagrama-de-gantt-de-un-proyecto}
\item Creately, herramienta online colaborativa. \\\url{https://creately.com/diagram/example/ieb3p3ml/LaTeX}
\item Se puede hacer en latex con el paquete \textit{pgfgantt}\\ \url{http://ctan.dcc.uchile.cl/graphics/pgf/contrib/pgfgantt/pgfgantt.pdf}
\end{itemize}

Pegar acá una captura de pantalla del diagrama de Gantt, cuidando que la letra sea suficientemente grande como para ser legible. 
Si el diagrama queda demasiado ancho, se puede pegar primero la ``tabla'' del Gantt y luego pegar la parte del diagrama de barras del diagrama de Gantt.

Configurar el software para que en la parte de la tabla muestre los códigos del EDT (WBS).\\
Configurar el software para que al lado de cada barra muestre el nombre de cada tarea.\\
Revisar que la fecha de finalización coincida con lo indicado en el Acta Constitutiva.

En la figura \ref{fig:gantt}, se muestra un ejemplo de diagrama de gantt realizado con el paquete de \textit{pgfgantt}. En la plantilla pueden ver el código que lo genera y usarlo de base para construir el propio.

\begin{figure}[htbp]
\begin{center}
\begin{ganttchart}{1}{12}
  \gantttitle{2020}{12} \\
  \gantttitlelist{1,...,12}{1} \\
  \ganttgroup{Group 1}{1}{7} \\
  \ganttbar{Task 1}{1}{2} \\
  \ganttlinkedbar{Task 2}{3}{7} \ganttnewline
  \ganttmilestone{Milestone o hito}{7} \ganttnewline
  \ganttbar{Final Task}{8}{12}
  \ganttlink{elem2}{elem3}
  \ganttlink{elem3}{elem4}
\end{ganttchart}
\end{center}
\caption{Diagrama de gantt de ejemplo}
\label{fig:gantt}
\end{figure}

\end{consigna}

\section{9. Matriz de uso de recursos de materiales}
\label{sec:recursos}


\begin{table}[htpb]
\label{tab:recursos}
\centering
\begin{tabularx}{\linewidth}{@{}|c|X|X|X|X|X|@{}}
\hline
\cellcolor[HTML]{C0C0C0} & \cellcolor[HTML]{C0C0C0} & \multicolumn{4}{c|}{\cellcolor[HTML]{C0C0C0}Recursos requeridos (horas)} \\ \cline{3-6} 
\multirow{-2}{*}{\cellcolor[HTML]{C0C0C0}\begin{tabular}[c]{@{}c@{}}Código\\ WBS\end{tabular}} & \multirow{-2}{*}{\cellcolor[HTML]{C0C0C0}\begin{tabular}[c]{@{}c@{}}Nombre \\ tarea\end{tabular}} & Material 1 & Material 2 & Material 3 & Material 4 \\ \hline
 &  &  &  &  &  \\ \hline
 &  &  &  &  &  \\ \hline
 &  &  &  &  &  \\ \hline
 &  &  &  &  &  \\ \hline
\end{tabularx}%
\end{table}


\section{10. Presupuesto detallado del proyecto}
\label{sec:presupuesto}

\begin{consigna}{red}
Si el proyecto es complejo entonces separarlo en partes:
\begin{itemize}
\item Un total global, indicando el subtotal acumulado por cada una de las áreas.
\item El desglose detallado del subtotal de cada una de las áreas.
\end{itemize}

IMPORTANTE: No olvidarse de considerar los COSTOS INDIRECTOS.

\end{consigna}

\begin{table}[htpb]
\centering
\begin{tabularx}{\linewidth}{@{}|X|c|r|r|@{}}
\hline
\rowcolor[HTML]{C0C0C0} 
\multicolumn{4}{|c|}{\cellcolor[HTML]{C0C0C0}COSTOS DIRECTOS} \\ \hline
\rowcolor[HTML]{C0C0C0} 
Descripción &
  \multicolumn{1}{c|}{\cellcolor[HTML]{C0C0C0}Cantidad} &
  \multicolumn{1}{c|}{\cellcolor[HTML]{C0C0C0}Valor unitario} &
  \multicolumn{1}{c|}{\cellcolor[HTML]{C0C0C0}Valor total} \\ \hline
 &
  \multicolumn{1}{c|}{} &
  \multicolumn{1}{c|}{} &
  \multicolumn{1}{c|}{} \\ \hline
 &
  \multicolumn{1}{c|}{} &
  \multicolumn{1}{c|}{} &
  \multicolumn{1}{c|}{} \\ \hline
\multicolumn{1}{|l|}{} &
   &
   &
   \\ \hline
\multicolumn{1}{|l|}{} &
   &
   &
   \\ \hline
\multicolumn{3}{|c|}{SUBTOTAL} &
  \multicolumn{1}{c|}{} \\ \hline
\rowcolor[HTML]{C0C0C0} 
\multicolumn{4}{|c|}{\cellcolor[HTML]{C0C0C0}COSTOS INDIRECTOS} \\ \hline
\rowcolor[HTML]{C0C0C0} 
Descripción &
  \multicolumn{1}{c|}{\cellcolor[HTML]{C0C0C0}Cantidad} &
  \multicolumn{1}{c|}{\cellcolor[HTML]{C0C0C0}Valor unitario} &
  \multicolumn{1}{c|}{\cellcolor[HTML]{C0C0C0}Valor total} \\ \hline
\multicolumn{1}{|l|}{} &
   &
   &
   \\ \hline
\multicolumn{1}{|l|}{} &
   &
   &
   \\ \hline
\multicolumn{1}{|l|}{} &
   &
   &
   \\ \hline
\multicolumn{3}{|c|}{SUBTOTAL} &
  \multicolumn{1}{c|}{} \\ \hline
\rowcolor[HTML]{C0C0C0}
\multicolumn{3}{|c|}{TOTAL} &
   \\ \hline
\end{tabularx}%
\end{table}


\section{11. Matriz de asignación de responsabilidades}
\label{sec:responsabilidades}
\begin{consigna}{red}
Establecer la matriz de asignación de responsabilidades y el manejo de la autoridad completando la siguiente tabla:

\begin{table}[htpb]
\centering
\resizebox{\textwidth}{!}{%
\begin{tabular}{|c|c|c|c|c|c|}
\hline
\rowcolor[HTML]{C0C0C0} 
\cellcolor[HTML]{C0C0C0} &
  \cellcolor[HTML]{C0C0C0} &
  \multicolumn{4}{c|}{\cellcolor[HTML]{C0C0C0}Listar todos los nombres y roles del proyecto} \\ \cline{3-6} 
\rowcolor[HTML]{C0C0C0} 
\cellcolor[HTML]{C0C0C0} &
  \cellcolor[HTML]{C0C0C0} &
  Responsable &
  Orientador &
  Equipo &
  Cliente \\ \cline{3-6} 
\rowcolor[HTML]{C0C0C0} 
\multirow{-3}{*}{\cellcolor[HTML]{C0C0C0}\begin{tabular}[c]{@{}c@{}}Código\\ WBS\end{tabular}} &
  \multirow{-3}{*}{\cellcolor[HTML]{C0C0C0}Nombre de la tarea} &
  \authorname &
  \supname &
  Nombre de alguien &
  \clientename \\ \hline
 &  &  &  &  &  \\ \hline
 &  &  &  &  &  \\ \hline
 &  &  &  &  &  \\ \hline
\end{tabular}%
}
\end{table}

{\footnotesize
Referencias:
\begin{itemize}
	\item P = Responsabilidad Primaria
	\item S = Responsabilidad Secundaria
	\item A = Aprobación
	\item I = Informado
	\item C = Consultado
\end{itemize}
} %footnotesize

Una de las columnas debe ser para el Director, ya que se supone que participará en el proyecto.
A su vez se debe cuidar que no queden muchas tareas seguidas sin ``A'' o ``I''.

Importante: es redundante poner ``I/A'' o ``I/C'', porque para aprobarlo o responder consultas primero la persona debe ser informada.

\end{consigna}

\section{12. Gestión de riesgos}
\label{sec:riesgos}

\begin{consigna}{red}
a) Identificación de los riesgos (al menos cinco) y estimación de sus consecuencias:
 
Riesgo 1: detallar el riesgo (riesgo es algo que si ocurre altera los planes previstos)
\begin{itemize}
\item Severidad (S): mientras más severo, más alto es el número (usar números del 1 al 10).\\
Justificar el motivo por el cual se asigna determinado número de severidad (S).
\item Probabilidad de ocurrencia (O): mientras más probable, más alto es el número (usar del 1 al 10).\\
Justificar el motivo por el cual se asigna determinado número de (O). 
\end{itemize}   

Riesgo 2:
\begin{itemize}
\item Severidad (S): 
\item Ocurrencia (O):
\end{itemize}

Riesgo 3:
\begin{itemize}
\item Severidad (S): 
\item Ocurrencia (O):
\end{itemize}


b) Tabla de gestión de riesgos:      (El RPN se calcula como RPN=SxO)

\begin{table}[htpb]
\centering
\begin{tabularx}{\linewidth}{@{}|X|c|c|c|c|c|c|@{}}
\hline
\rowcolor[HTML]{C0C0C0} 
Riesgo & S & O & RPN & S* & O* & RPN* \\ \hline
       &   &   &     &    &    &      \\ \hline
       &   &   &     &    &    &      \\ \hline
       &   &   &     &    &    &      \\ \hline
       &   &   &     &    &    &      \\ \hline
       &   &   &     &    &    &      \\ \hline
\end{tabularx}%
\end{table}

Criterio adoptado: 
Se tomarán medidas de mitigación en los riesgos cuyos números de RPN sean mayores a ....

Nota: los valores marcados con (*) en la tabla corresponden luego de haber aplicado la mitigación.

c) Plan de mitigación de los riesgos que originalmente excedían el RPN máximo establecido:
 
Riesgo 1: Plan de mitigación (si por el RPN fuera necesario elaborar un plan de mitigación).
  Nueva asignación de S y O, con su respectiva justificación:
  - Severidad (S): mientras más severo, más alto es el número (usar números del 1 al 10).
          Justificar el motivo por el cual se asigna determinado número de severidad (S).
  - Probabilidad de ocurrencia (O): mientras más probable, más alto es el número (usar del 1 al 10).
          Justificar el motivo por el cual se asigna determinado número de (O).

Riesgo 2: Plan de mitigación (si por el RPN fuera necesario elaborar un plan de mitigación).
 
Riesgo 3: Plan de mitigación (si por el RPN fuera necesario elaborar un plan de mitigación)

\end{consigna}


\section{13. Gestión de la calidad}
\label{sec:calidad}

\begin{consigna}{red}
Para cada uno de los requerimientos del proyecto indique:
\begin{itemize} 
\item Req \#1: Copiar acá el requerimiento.

Verificación y validación:

\begin{itemize}
\item Verificación para confirmar si se cumplió con lo requerido antes de mostrar el sistema al cliente:\\
Detallar 
\item Validación con el cliente para confirmar que está de acuerdo en que se cumplió con lo requerido:\\
Detallar  
\end{itemize}

\end{itemize}

Tener en cuenta que en este contexto se pueden mencionar simulaciones, cálculos, revisión de hojas de datos, consulta con expertos, etc.

\end{consigna}

\section{14. Comunicación del proyecto}
\label{sec:comunicaciones}

\begin{consigna}{red}
El plan de comunicación del proyecto es el siguiente:
\end{consigna}

% Please add the following required packages to your document preamble:
% \usepackage{graphicx}
% \usepackage[table,xcdraw]{xcolor}
% If you use beamer only pass "xcolor=table" option, i.e. \documentclass[xcolor=table]{beamer}
\begin{table}[htpb]
\centering
\resizebox{\textwidth}{!}{%
\begin{tabular}{|c|c|c|c|c|c|}
\hline
\rowcolor[HTML]{C0C0C0} 
\multicolumn{6}{|c|}{\cellcolor[HTML]{C0C0C0}PLAN DE COMUNICACIÓN DEL PROYECTO}           \\ \hline
\rowcolor[HTML]{C0C0C0} 
¿Qué comunicar? & Audiencia & Propósito & Frecuencia & Método de comunicac. & Responsable \\ \hline
                &           &           &            &                      &             \\ \hline
                &           &           &            &                      &             \\ \hline
                &           &           &            &                      &             \\ \hline
                &           &           &            &                      &             \\ \hline
                &           &           &            &                      &             \\ \hline
\end{tabular}%
}
\end{table}

\section{15. Gestión de Compras}
\label{sec:compras}

\begin{consigna}{red}
En caso de tener que comprar elementos o contratar servicios:
a) Explique con qué criterios elegiría a un proveedor.
b) Redacte el Statement of Work correspondiente.
\end{consigna}

\section{16. Seguimiento y control}
\label{sec:seguimiento}

\begin{consigna}{red}
Para cada tarea del proyecto establecer la frecuencia y los indicadores con los se seguirá su avance y quién será el responsable de hacer dicho seguimiento y a quién debe comunicarse la situación (en concordancia con el Plan de Comunicación del proyecto).

El indicador de avance tiene que ser algo medible, mejor incluso si se puede medir en \% de avance. Por ejemplo,se pueden indicar en esta columna cosas como ``cantidad de conexiones ruteadeas'' o ``cantidad de funciones implementadas'', pero no algo genérico y ambiguo como ``\%'', porque el lector no sabe porcentaje de qué cosa.

\end{consigna}

\begin{table}[!htpb]
\centering
\begin{tabularx}{\linewidth}{@{}|X|X|X|X|X|X|@{}}
\hline
\rowcolor[HTML]{C0C0C0} 
\multicolumn{6}{|c|}{\cellcolor[HTML]{C0C0C0}SEGUIMIENTO DE AVANCE}                                                                       \\ \hline
\rowcolor[HTML]{C0C0C0} 
Tarea del WBS & Indicador de avance & Frecuencia de reporte & Resp. de seguimiento & Persona a ser informada & Método de comunic. \\ \hline
 &  &  &  &  &  \\ \hline
 &  &  &  &  &  \\ \hline
 &  &  &  &  &  \\ \hline
 &  &  &  &  &  \\ \hline
 &  &  &  &  &  \\ \hline
\end{tabularx}%
%}
\end{table}

\section{17. Procesos de cierre}    
\label{sec:cierre}

\begin{consigna}{red}
Establecer las pautas de trabajo para realizar una reunión final de evaluación del proyecto, tal que contemple las siguientes actividades:

\begin{itemize}
\item Pautas de trabajo que se seguirán para analizar si se respetó el Plan de Proyecto original:
 - Indicar quién se ocupará de hacer esto y cuál será el procedimiento a aplicar. 
\item Identificación de las técnicas y procedimientos útiles e inútiles que se utilizaron, y los problemas que surgieron y cómo se solucionaron:
 - Indicar quién se ocupará de hacer esto y cuál será el procedimiento para dejar registro.
\item Indicar quién organizará el acto de agradecimiento a todos los interesados, y en especial al equipo de trabajo y colaboradores:
  - Indicar esto y quién financiará los gastos correspondientes.
\end{itemize}

\end{consigna}


\end{document}
